\section{D. Irga B. Naufal Fakhri}
\subsection{Teori}
\begin{enumerate}
\item Jenis-jenis variabel pada python dan cara penggunaannya:

\begin{enumerate}
\item Boolean
\lstinputlisting[caption=Contoh kode variable Boolean., firstline=8, lastline=10]{src/2/1174066/1174066.py}

\item String
\lstinputlisting[caption=Contoh kode variable String., firstline=12, lastline=14]{src/2/1174066/1174066.py}

\item Integer
\lstinputlisting[caption=Contoh kode variable Integer., firstline=16, lastline=18]{src/2/1174066/1174066.py}

\item Float
\lstinputlisting[caption=Contoh kode variable Float., firstline=20, lastline=22]{src/2/1174066/1174066.py}

\item Hexadecimal
\lstinputlisting[caption=Contoh kode variable Hexadecimal., firstline=24, lastline=26]{src/2/1174066/1174066.py}

\item Complex
\lstinputlisting[caption=Contoh kode variable Complex., firstline=28, lastline=30]{src/2/1174066/1174066.py}

\item List
\lstinputlisting[caption=Contoh kode variable List., firstline=32, lastline=35]{src/2/1174066/1174066.py}

\item Tuple
\lstinputlisting[caption=Contoh kode variable Tuple., firstline=37, lastline=40]{src/2/1174066/1174066.py}

\item Set
\lstinputlisting[caption=Contoh kode variable Set., firstline=42, lastline=44]{src/2/1174066/1174066.py}

\item Dictionary
\lstinputlisting[caption=Contoh kode variable Dictionary., firstline=46, lastline=49]{src/2/1174066/1174066.py}

\end{enumerate}

\item Permintaan Input dari user dan Outputnya
\lstinputlisting[caption=Contoh kode input dan outputnya., firstline=51, lastline=53]{src/2/1174066/1174066.py}

\item Operator dasar aritmatika dan perubahan tipe data variable

Operator dasar aritmatika
\begin{enumerate}
\item Perjumlahan (+)
Operator ini berfungsi untuk melakukan operasi perjumlahan.
\lstinputlisting[caption=Contoh kode operasi pertambahan., firstline=51, lastline=60]{src/2/1174066/1174066.py}
\item Pengurangan (-)
Operator ini berfungsi untuk melakukan operasi pengurangan.
\lstinputlisting[caption=Contoh kode operasi pengurangan., firstline=62, lastline=66]{src/2/1174066/1174066.py}
\item Perkalian (*)
Operator ini dipergunakan untuk melakukan operasi perkalian.
\lstinputlisting[caption=Contoh kode operasi perkalian., firstline=68, lastline=72]{src/2/1174066/1174066.py}
\item Pembagian (/)
Operator ini dipergunakan untuk melakukan operasi pembagian.
\lstinputlisting[caption=Contoh kode operasi pembagian., firstline=74, lastline=78]{src/2/1174066/1174066.py}
\item Modulus (%)
Operator ini dipergunakan untuk melakukan operasi modulus.
\lstinputlisting[caption=Contoh kode operasi modulus., firstline=80, lastline=84]{src/2/1174066/1174066.py}
\item Perpangkatan (**)
Operator ini dipergunakan untuk melakukan operasi perpangkatan.
\lstinputlisting[caption=Contoh kode operasi perpangkatan., firstline=86, lastline=90]{src/2/1174066/1174066.py}
\item Pembulatan Kebawah Pada Hasil Pembagian (//)
Operator ini dipergunakan untuk melakukan operasi pembulatan hasil bagi.
\lstinputlisting[caption=Contoh kode operasi pembulatan hasil pembagian kebawah., firstline=92, lastline=96]{src/2/1174066/1174066.py}
\end{enumerate}

Perubahan tipe data variable
\begin{enumerate}
\item String menjadi Integer
\lstinputlisting[caption=Contoh kode variable string menjadi integer., firstline=99, lastline=102]{src/2/1174066/1174066.py}
\item Integer menjadi String
\lstinputlisting[caption=Contoh kode variable integer menjadi string., firstline=104, lastline=107]{src/2/1174066/1174066.py}
\end{enumerate}


\item Sintak perulangan (looping), jenis-jenisnya, dan penggunaannya.
\begin{enumerate}
\item While Loop
While Loop adalah perulangan yang mengeksekusi statement terus menerus selama kondisi bernilai True.
\lstinputlisting[caption=Contoh kode penggunaan while loop., firstline=111, lastline=115]{src/2/1174066/1174066.py}

\item For Loop
For Loop  adalah pengulangan berdasarkan kondisi yang telah ditentukan biasanya kondisi pertambahan seperti 1 sampai 5
\lstinputlisting[caption=Contoh kode penggunaan for loop., firstline=117, lastline=120]{src/2/1174066/1174066.py}

\item Nested Loop
Nested Loop merupakan pengulangan yang ada di dalam pengulangan
\lstinputlisting[caption=Contoh kode penggunaan nested loop., firstline=122, lastline=129]{src/2/1174066/1174066.py}

\end{enumerate}

\item Sintak kondisi dan penggunaannya.
\begin{enumerate}
\item If
Kondisi ini digunakan untuk mengecek apabila kondisi tersebut dipenuhi akan mengeksekusi kode didalamnya.
\lstinputlisting[caption=Contoh kode penggunaan if., firstline=132, lastline=136]{src/2/1174066/1174066.py}

\item If Else
Kondisi ini digunakan untuk mengecek apabila kondisi tersebut dipenuhi akan mengeksekusi kode didalamnya dan didalamnya memiliki dua kondisi.
\lstinputlisting[caption=Contoh kode penggunaan if else., firstline=138, lastline=143]{src/2/1174066/1174066.py}

\item Elif
Kondisi ini digunakan untuk mengecek apabila kondisi tersebut dipenuhi akan mengeksekusi kode didalamnya dan didalamnya memiliki dua kondisi atau lebih.
\lstinputlisting[caption=Contoh kode penggunaan elif., firstline=145, lastline=152]{src/2/1174066/1174066.py}

\item Kondisi di dalam kondisi
Kondisi ini digunakan saat kondisi memerlukan kondisi lagi didalamnya
\lstinputlisting[caption=Contoh kode penggunaan kondisi di dalam kondisi., firstline=155, lastline=165]{src/2/1174066/1174066.py}

\end{enumerate}

\item Jenis-jenis error pada python dan cara mengatasinya.
\begin{itemize}
\item Syntax Errors
Syntax Errors adalah kesalahan pada penulisan syntax atau kode. Solusinya adalah memperbaiki penulisan syntax atau kode

\item Zero Division Error
ZeroDivisonError adalah exceptions yang terjadi saat eksekusi program menghasilkan perhitungan matematika pembagian dengan angka nol (0). Solusinya adalah tidak membagi suatu yang hasilnya nol.

\item Name Error
NameError adalah exception saat kode melakukan eksekusi terhadap local name atau global name yang tidak terdefinisi atau tidak ada. Solusinya adalah memastikan variabel atau function yang akan dipanggil ada didalam program atau tidak salah mengetikannya.

\item Type Error
TypeError adalah exception saat melakukan eksekusi terhadap suatu operasi atau fungsi dengan type object yang tidak sesuai. Solusinya adalah mengkoversi varibelnya sesuai dengan tipe data sesuai dengan yang akan digunakan.

\end{itemize}

\item Cara pemakaian Try Except.
\lstinputlisting[caption=Contoh kode penggunaan try except., firstline=168, lastline=174]{src/2/1174066/1174066.py}

\end{enumerate}