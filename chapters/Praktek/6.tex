\section{D. Irga B. Naufal Fakhri}
\subsection{Buatlah library fungsi (file terpisah/library dengan nama NPM/textunderscore bar.py) untuk plot dengan jumlah subplot adalah NPM mod 3 + 2}

Ini adalah fungsi untuk membuat plot bar sesuai dengan hasil modulus:
\lstinputlisting[firstline=8, lastline=44]{src/6/1174066/Praktek/l1174066_bar.py}

Dan ini adalah cara memanggilnya:
\lstinputlisting[firstline=7, lastline=21]{src/6/1174066/Praktek/main.py}

\subsection{Buatlah library fungsi (file terpisah/library dengan nama NPM/textunderscore scatter.py) untuk plot dengan jumlah subplot adalah NPM mod 3 + 2}

Ini adalah fungsi untuk membuat plot scatter sesuai dengan hasil modulus:
\lstinputlisting[firstline=8, lastline=44]{src/6/1174066/Praktek/l1174066_scatter.py}

Dan ini adalah cara memanggilnya:
\lstinputlisting[firstline=23, lastline=37]{src/6/1174066/Praktek/main.py}

\subsection{Buatlah library fungsi (file terpisah/library dengan nama NPM/textunderscore pie.py) untuk plot dengan jumlah subplot adalah NPM mod 3 + 2}

Ini adalah fungsi untuk membuat plot pie sesuai dengan hasil modulus:
\lstinputlisting[firstline=8, lastline=50]{src/6/1174066/Praktek/l1174066_pie.py}

Dan ini adalah cara memanggilnya:
\lstinputlisting[firstline=39, lastline=53]{src/6/1174066/Praktek/main.py}

\subsection{Buatlah library fungsi (file terpisah/library dengan nama NPM/textunderscore pie.py) untuk plot dengan jumlah subplot adalah NPM mod 3 + 2}

Ini adalah fungsi untuk membuat subplot sesuai dengan hasil modulus:
\lstinputlisting[firstline=8, lastline=44]{src/6/1174066/Praktek/l1174066_plot.py}

Dan ini adalah cara memanggilnya:
\lstinputlisting[firstline=55, lastline=69]{src/6/1174066/Praktek/main.py}


\subsection{Keterampilan Penanganan Error}
\lstinputlisting[firstline=7, lastline=19]{src/6/1174066/Praktek/1174066_error.py}