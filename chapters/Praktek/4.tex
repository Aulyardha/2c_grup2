\section{D. Irga B. Naufal Fakhri}
\subsection{Soal 1}
Buatlah fungsi untuk membuka file csv dengan lib csv mode list
\lstinputlisting[firstline=9, lastline=20]{src/4/1174066/Praktek/lib_1174066_csv.py}

\subsection{Soal 2}
Buatlah fungsi untuk membuka file csv dengan lib csv mode dictionary
\lstinputlisting[firstline=24, lastline=34]{src/4/1174066/Praktek/lib_1174066_csv.py}

\subsection{Soal 3}
Buatlah fungsi untuk membuka file csv dengan lib pandas mode list
\lstinputlisting[firstline=9, lastline=11]{src/4/1174066/Praktek/lib_1174066_pandas.py}

\subsection{Soal 4}
Buatlah fungsi untuk membuka file csv dengan lib pandas mode dictionary
\lstinputlisting[firstline=15, lastline=19]{src/4/1174066/Praktek/lib_1174066_pandas.py}

\subsection{Soal 5}
Buat fungsi baru untuk mengubah format tanggal menjadi standar dataframe
\lstinputlisting[firstline=22, lastline=24]{src/4/1174066/Praktek/lib_1174066_pandas.py}

\subsection{Soal 6}
Buat fungsi baru  untuk mengubah index kolom
\lstinputlisting[firstline=28, lastline=30]{src/4/1174066/Praktek/lib_1174066_pandas.py}

\subsection{Soal 7}
Buat fungsi baru untuk mengubah atribut atau nama kolom
\lstinputlisting[firstline=35, lastline=37]{src/4/1174066/Praktek/lib_1174066_pandas.py}

\subsection{Soal 8}
Buat program main yang menggunakan library NPM csv yang membuat dan membaca file csv
\lstinputlisting[caption=lib\textunderscore 1174066\textunderscore csv.py, firstline=38, lastline=43]{src/4/1174066/Praktek/lib_1174066_csv.py}

\lstinputlisting[caption=main.py, firstline=8, lastline=17]{src/4/1174066/Praktek/main.py}

\subsection{Soal 9}
Buat program main2.py yang menggunakan library NPM pandas.py yang membuat dan membaca file csv
\lstinputlisting[caption=main2.py, firstline=42, lastline=48]{src/4/1174066/Praktek/lib_1174066_pandas.py}
\lstinputlisting[caption=main2.py, firstline=8, lastline=17]{src/4/1174066/Praktek/main2.py}

\subsection{Keterampilan Penanganan Error}
Pada praktikum saat ini saya tidak mendapatkan error

%%%%%%%%%%%%%%%%%%%%%%%%%%%%%%%%%%%%%%%%%%%%%%%%%%%%%%%%%%%%%%%%%%%%%%%%%%%%%%%%%%%%%%%%%%%%%%%%%%%%%%%%%%%%%%%%%%%%%%%%%%
\section{Fanny Shafira Damayanti | 1174069}
\subsection{Keterampilan Pemrograman}
\begin{enumerate}
	\item NO 1 

	\lstinputlisting[firstline=10, lastline=15]{src/4/1174069/Praktek/1174069_csv.py}

	\item NO 2

	\lstinputlisting[firstline=17, lastline=22]{src/4/1174069/Praktek/1174069_csv.py}

	\item NO 3 

	\lstinputlisting[firstline=10, lastline=13]{src/4/1174069/Praktek/1174069_pandas.py}

	\item NO 4 

	\lstinputlisting[firstline=10, lastline=13]{src/4/1174069/Praktek/1174069_pandas.py}

	\item NO 5 

	\lstinputlisting[firstline=15, lastline=19]{src/4/1174069/Praktek/1174069_pandas.py}

	\item NO 6 

	\lstinputlisting[firstline=21, lastline=24]{src/4/1174069/Praktek/1174069_pandas.py}

	\item NO 7 

	\lstinputlisting[firstline=26, lastline=30]{src/4/1174069/Praktek/1174069_pandas.py}

	\item NO 8

	\lstinputlisting[firstline=8, lastline=13]{src/4/1174069/Praktek/main.py}

	\item NO 9 

	\lstinputlisting[firstline=8, lastline=13]{src/4/1174069/Praktek/main2.py}

\end{enumerate}

\subsection{Penanganan Error}
\begin{enumerate}
	\item Peringatan error yang terdapat pada praktikum chapter 4 ini yaitu :

	\begin{itemize}
		\item Syntax Errors
		Syntax Errors terjadi ketika ada kesalahan dalam meuliskan kode. Solusinya adalah memperbaiki penulisan kode yang salah.

		\item Name Error
		NameError terjadi ketika salah mengetikan kode local name yang tidak terdefinisi. Solusinya adalah menuliskan kode dengan benar agar function nya dapat terpanggil. 

		\item Type Error
		TypeError terjadi pada saat eksekusi terhadapt fungsi dengan tipe objek tidak sesuai. Solusinya mengkonversi variablenya harus sesuai dengan tipe datanya.
	\end{itemize}

	Contoh Penggunaan TryExcept
	\lstinputlisting[firstline=55, lastline=67]{src/4/1174069/Praktek/1174069.py}
\end{enumerate}

%%%%%%%%%%%%%%%%%%%%%%%%%%%%%%%%%%%%%%%%%%%%%%%%%%%%%%%%%%%%%%%%%%%%%%%%%%%%%%%%%%%%%%%%%%%%%%%%%%%%%%%%%%%%%%%%%%%%
\section{Aulyardha Anindita | 1174054}
\subsection{Keterampilan Pemrograman}
\begin{enumerate}

\item Jawaban Soal No. 1
\lstinputlisting[firstline=10, lastline=15]{src/4/1174054/Praktek/1174054csv.py}

\item Jawaban Soal No. 2
\lstinputlisting[firstline=17, lastline=22]{src/4/1174054/Praktek/1174054csv.py}

\item Jawaban Soal No. 3
\lstinputlisting[firstline=10, lastline=13]{src/4/1174054/Praktek/1174054pandas.py}

\item Jawaban Soal No. 4
\lstinputlisting[firstline=10, lastline=13]{src/4/1174054/Praktek/1174054pandas.py}

\item Jawaban Soal No. 5
\lstinputlisting[firstline=15, lastline=19]{src/4/1174054/Praktek/1174054pandas.py}

\item Jawaban Soal No. 6
\lstinputlisting[firstline=21, lastline=24]{src/4/1174054/Praktek/1174054pandas.py}

\item Jawaban Soal No. 7
\lstinputlisting[firstline=26, lastline=30]{src/4/1174054/Praktek/1174054pandas.py}

\item Jawaban Soal No. 8
\lstinputlisting[firstline=8, lastline=13]{src/4/1174054/Praktek/main.py}

\item Jawaban Soal No. 9
\lstinputlisting[firstline=8, lastline=13]{src/4/1174054/Praktek/main2.py}

\subsection{Keterampilan Penanganan Error}

Peringatan error di praktek keempat ini, yaitu:
\begin{itemize}
\item Syntax Errors
Syntax Errors adalah keadaan dimana pada kode python mnengalami kesalahan dalam penulisan. Untuk mengatasinya yaitu dengan memperbaiki penulisan kode yang salah 

\item Name Error
NameError adalah suatu keadaan atau exception yang terjadi ketika kode melakukan eksekusi terhadap local name atau global name yang tidak terdefinisi. Untuk mengatasinya yaitu dengan memastikan variabel atau function yang dipanggil ada atau tidak salah ketik.

\item Type Error
TypeError adalah suatu keadaan atau exception yang akan terjadi apabila pada saat dilakukannya eksekusi terhadap suatu operasi atau fungsi dengan type object yang tidak sesuai. Untuk mengatasinya yaitu dengan mengkoversi varibelnya sesuai dengan tipe data yang akan digunakan.
\end{itemize}

Fungsi yang menggunakan try except
\lstinputlisting[firstline=55, lastline=67]{src/4/1174054/Praktek/1174054.py}

\end{enumerate}

%%%%%%%%%%%%%%%%%%%%%%%%%%%%%%%%%%%%%%%%%%%%%%%%%%%%%%%%%%%%%%%%%%%%%%

\section{Nurul Izza Hamka | 1174062 | Teori}
\section{Keterampilan Pemrograman}
\begin{enumerate}

\item NO1

\lstinputlisting[firstline=10, lastline=15]{src/4/1174062/Praktek/1174062csv.py}

\item NO2

\lstinputlisting[firstline=17, lastline=22]{src/4/1174062/Praktek/1174062csv.py}

\item NO3
	
\lstinputlisting[firstline=10, lastline=13]{src/4/1174062/Praktek/1174062Pandas.py}

\item NO4

\lstinputlisting[firstline=10, lastline=13]{src/4/1174062/Praktek/1174062Pandas.py}

\item NO5

\lstinputlisting[firstline=15, lastline=19]{src/4/1174062/Praktek/1174062Pandas.py}

\item NO6

\lstinputlisting[firstline=21, lastline=24]{src/4/1174062/Praktek/1174062Pandas.py}

\item NO7

\lstinputlisting[firstline=26, lastline=30]{src/4/1174062/Praktek/1174062Pandas.py}

\item NO8

\lstinputlisting[firstline=8, lastline=13]{src/4/1174062/Praktek/main.py}

\item NO9

\lstinputlisting[firstline=8, lastline=13]{src/4/1174062/Praktek/main2.py}
\end{enumerate}
\section{Ketrampilan Penanganan Error}

\begin{itemize}
\item NameError
Error yang terjadi adalah NameError yang mana terjadi kesalahan pada saat melakukan eksekusi dan tidak dapat terdefinisi.
Penanganan yang dapat dilakukan adalah memastikan bahwa Variable dan Function yang akan kita panggil pastikan semua benar.

\item SyntaxError
Tipe Error ini adalah ada kesalahan pada penulisan, untuk itu pastikan penulisannya benar.

\item TypeError
Error ini terjadi pada saat eksekusi terhadap dan operasi dan fungsi masuk kedalam objek sedangkan typenya salah.

\item Menggunakan Try Except 

\lstinputlisting[firstline=55, lastline=67]{src/4/1174062/Praktek/1174062.py}
\end{itemize}

%%%%%%%%%%%%%%%%%%%%%%%%%%%%%%%%%%%%%%%%%%%%%%%%%%%%%%%%%%%%
\section{Chandra Kirana Poetra}

\subsection{Soal 1}
Buatlah fungsi (file terpisah/library dengan nama NPM csv.py) untuk membuka file csv dengan lib csv mode list
\lstinputlisting[firstline=10, lastline=21]{src/4/1174079/Praktek/lib_1174079_csv.py}

\subsection{Soal 2}
Buatlah fungsi (file terpisah/library dengan nama NPM csv.py) untuk membuka file csv dengan lib csv mode dictionary
\lstinputlisting[firstline=23, lastline=33]{src/4/1174079/Praktek/lib_1174079_csv.py}

\subsection{Soal 3}
Buatlah fungsi (file terpisah/library dengan nama NPM csv.py) untuk membuka file csv dengan lib csv mode dictionary
\lstinputlisting[firstline=11, lastline=13]{src/4/1174079/Praktek/lib_1174079_pandas.py}

\subsection{Soal 4}
Buatlah fungsi (file terpisah/library dengan nama NPM csv.py) untuk membuka file csv dengan lib csv mode dictionary
\lstinputlisting[firstline=15, lastline=17]{src/4/1174079/Praktek/lib_1174079_pandas.py}

\subsection{Soal 5}
Buat fungsi baru di NPM pandas.py untuk mengubah format tanggal menjadi
standar dataframe
\lstinputlisting[firstline=19, lastline=21]{src/4/1174079/Praktek/lib_1174079_pandas.py}

\subsection{Soal 6}
Buat fungsi baru di NPM pandas.py untuk mengubah index kolom
\lstinputlisting[firstline=23, lastline=25]{src/4/1174079/Praktek/lib_1174079_pandas.py}

\subsection{Soal 7}
Buat fungsi baru di NPM pandas.py untuk mengubah atribut atau nama kolom
\lstinputlisting[firstline=27, lastline=29]{src/4/1174079/Praktek/lib_1174079_pandas.py}

\subsection{Soal 8}
Buat program main.py yang menggunakan library NPM csv.py yang membuat
dan membaca file csv
\lstinputlisting[firstline=35, lastline=40]{src/4/1174079/Praktek/lib_1174079_csv.py}

\lstinputlisting[caption=main.py, firstline=8, lastline=15]{src/4/1174079/Praktek/main.py}

\subsection{Soal 9}
Buat program main2.py yang menggunakan library NPM pandas.py yang membuat dan membaca file csv

\lstinputlisting[caption=main2.py, firstline=31, lastline=37]{src/4/1174079/Praktek/lib_1174079_pandas.py}
\lstinputlisting[caption=main2.py, firstline=8, lastline=15]{src/4/1174079/Praktek/main2.py}

\subsection{Keterampilan Penanganan Error}
Unicode Decode Error
Terjadi ketika kesalahan decode untuk unicode terdeteksi, typo semacamnya

\lstinputlisting[firstline=10, lastline=21]{src/4/1174079/Praktek/contoherrorpandas.py}




